
\usetikzlibrary{shapes,arrows}

\newcommand{\hoo}[0]{H_2O}

\tikzstyle{decision} = [diamond, draw, fill=blue!30, text badly centered, node distance=3.5cm, inner sep=3pt]
    
    %  text width=15em
\tikzstyle{block} = [inner sep = 10,rectangle, draw, fill=black!10, text centered, rounded corners, minimum height=5em,node distance=2.5cm]
    
\tikzstyle{line} = [draw, -latex']

\tikzstyle{cloud} = [draw, rectangle,fill=orange!40, node distance=5cm,
    minimum height=2em]

\begin{tikzpicture}
%[node distance = 1cm, auto]



% 
% \node [block] (pcond) {Speculate $p_{cond} = 0.05\, bar$};
% \node [block, below of=pcond,align=center] (eval_turbine) {Compute point 2 and 3\\ enthalpy from $p_{cond} $};
% 
% \node [right of=eval_turbine,node distance=8cm] (null1) {};
% 
% \node [block, below of=eval_turbine] (qdot_cond) {$\dot{Q}_{cond} = \dot{m}_{ST} \cdot (h_2-h_3)$};
% \node [block, below of=qdot_cond] (Tout) {$\displaystyle \Delta T_{H_2O}= \frac{ \dot{Q}_{cond} }{ \dot{m}_{H_2O} \cdot c_{p_{H_2O}} }$ };
% \node [block, below of=Tout] (w_water) {$\displaystyle  w_{H_2O}= \frac{ \dot{m}_{H_2O}  }{ \overline{\rho}_{H_2O} \cdot A_{IN} \cdot N_{tubes} }$ };
% 
% %suggestion
% \node [cloud, left of=w_water,align=center] (rho_water) {$\overline{\rho}_{H_2O}$ is a mean value\\ between $T_{H_2O}^{IN}$ and $T_{H_2O}^{OUT}$};
% \node [cloud, right of=w_water,align=center] (Ain) {$A_{IN}$ is the section \\ of each tube};
% 
% \node [block, below of=w_water] (mu) {$\displaystyle \mu_{H_2O} = \mu \left( \frac{ T_{H_2O}^{IN} + T_{H_2O}^{OUT} }{ 2 } \right)$ };
% \node [block, below of=mu,align=center] (nu) {Find $Re$ and $Pr$ and \\ from that compute $Nu$ };
% 
% %suggestion
% \node [cloud, left of=nu,align=center] (re) {$\displaystyle Re = \frac{\rho_{\hoo} \cdot w_{\hoo} \cdot D_{int}}{\mu_{\hoo}}$};
% \node [cloud, right of=nu,align=center] (pr) {$\displaystyle Pr = \frac{\mu_{\hoo} \cdot c_{p_{\hoo}}}{k_{\hoo}}$};
% 
% \node [block, below of=nu,align=center] (U) {From $nu \quad \Rightarrow  \quad h_{INT}$ \\ From $ h_{INT}  \quad \Rightarrow  \quad U$ };
% \node [block, below of=U,align=center] (NTU) {$\displaystyle NTU = \frac{ U\cdot A_{TOT} }{ \dot{m}_{H_2O} \cdot c_{p_{H_2O}} }$};
% \node [block, below of=NTU,align=center] (eps) {$\displaystyle \varepsilon = 1- e^{-NTU} $};
% \node [block, below of=eps,align=center] (tcond) {$\displaystyle T_{cond} = T_{H_2O}^{IN} + \frac{ T_{H_2O}^{OUT} - T_{H_2O}^{IN} }{ \varepsilon  }$};
% \node [block, below of=tcond,align=center] (pcond_new) {$\displaystyle p_{cond} = p_{sat}(T_{cond})$};
% 
% \node [decision, below of=pcond_new,align=center] (decide) {$| p_{cond} - p_{cond}^{old} | $\\ $<$ \\tollerance};
%     
% \node [block,node distance=7cm,fill=green!40, left of=decide,align=center] (yes) {Solution found!};
% \node [block,node distance=8cm,fill=red!40, right of=decide,align=center] (no) {Error is too large,\\ another loop is needed};
% 
% % arrows
% \path [line] (pcond) -- (eval_turbine);
% \path [line] (eval_turbine) -- (qdot_cond);
% \path [line] (qdot_cond) -- (Tout);
% \path [line] (Tout) -- (w_water);
% 
% \path [line] (rho_water) -- (w_water) ;
% \path [line] (Ain) -- (w_water) ;
% 
% \path [line] (w_water) -- (mu);
% \path [line] (mu) -- (nu);
% 
% \path [line] (re) -- (nu) ;
% \path [line] (pr) -- (nu) ;
% 
% \path [line] (nu) -- (U);
% \path [line] (U) -- (NTU);
% \path [line] (NTU) -- (eps);
% \path [line] (eps) -- (tcond);
% \path [line] (tcond) -- (pcond_new);
% \path [line] (pcond_new) -- (decide);
% 
% \path [line] (decide) -- node[anchor=south] {yes} (yes);
% 
% \path [line] (decide) -- node[anchor=south] {no} (no);
% 
% \path [line] (no) |- (eval_turbine) ;
%\path [line] (decide) -| node[near start,anchor=south] {no} (null1) -- (eval_turbine) ;
%\path [line] (-3,0) --++(pcond_new) -| (eval_turbine);


\node [block,align=center] (start) {Define an initial value for\\
the size of the crack, $a_0$, $c_0$ };

\node [block, align=center, below of = start] (deltaK) {Find $\Delta K_a$ and $\Delta K_c$ function of the\\ 
previous size of the crack  $a_i$ and $c_i$};

\node [block, align=center, below of = deltaK, node distance=3cm] (dadN) {$\begin{aligned}
\text{Calculate} \  \left. \dfrac{da}{dN}\right\lvert_i &= \text{C} \, \Delta K_a ^m \\
\text{and} \ \left. \dfrac{dc}{dN}\right\lvert_i &= \text{C} \, \Delta K_c ^m \\
\end{aligned}$};

\node [block, align=center, below of = dadN, node distance=3.5cm] (update) {Update the crack size with Paris'model:\\
$\begin{aligned}
a_{i+1} &= \left. \dfrac{da}{dN}\right\lvert_i \cdot \Delta N =  \text{C} \, \Delta K_a ^m  \cdot \Delta N \\
c_{i+1} &=\left. \dfrac{dc}{dN}\right\lvert_i \cdot \Delta N  = \text{C} \, \Delta K_c ^m  \cdot \Delta N
\end{aligned}$};

\node [decision, below of=update,align=center, node distance=4.cm] (decide) {$a_{i+1} > a_{limit} $\\ or \\ $\Delta K_{max} > K_{IC}$};
    
\node [block,node distance=7cm,fill=red!40, left of=decide,align=center] (yes) {The vessel fails.};
\node [block,node distance=7.5cm,fill=green!40!olive!60, right of=decide,align=center] (no) {The crack can propagate more.};

\node [cloud, right of=dadN,align=center] (c_and_m) {C and m are the\\ Paris'law coefficients.};
\node [cloud, left of=update,align=center, node distance=6.5cm] (N) {
$\Delta N$ is  equal to 1\\
to evaluate the crack\\
propagation at every cycle.};

% now we link together the blocks with arrows

\path [line] (start) -- (deltaK);
\path [line]  (deltaK) -- (dadN);
\path [line]   (dadN) -- (update);
\path [line] (update)--(decide);

\path [line] (decide) -- node[anchor=south] {yes} (yes);
\path [line] (decide) -- node[anchor=south] {no} (no);

\path [line] (no) |- (deltaK) ;

\path[line] (c_and_m) --  (dadN) ;
\path[line] (N) --  (update) ;

\end{tikzpicture}